\documentclass{article}
\usepackage[utf8]{inputenc}
\usepackage{listings}
\usepackage{xcolor}
\usepackage{hyperref}
\usepackage[a4paper, total={6in, 8in}]{geometry}

\title{Behind the Scenes}
\author{Carlo Ramponi}
\date{22 April 2022}

\begin{document}

\definecolor{codegreen}{rgb}{0,0.6,0}
\definecolor{codegray}{rgb}{0.5,0.5,0.5}
\definecolor{codepurple}{rgb}{0.58,0,0.82}
\definecolor{backcolour}{rgb}{0.95,0.95,0.92}

\lstdefinestyle{mystyle}{
    backgroundcolor=\color{backcolour},   
    commentstyle=\color{codegreen},
    keywordstyle=\color{magenta},
    numberstyle=\tiny\color{codegray},
    stringstyle=\color{codepurple},
    basicstyle=\ttfamily\footnotesize,
    breakatwhitespace=false,         
    breaklines=true,                 
    captionpos=b,                    
    keepspaces=true,                 
    numbers=left,                    
    numbersep=5pt,                  
    showspaces=false,                
    showstringspaces=false,
    showtabs=false,                  
    tabsize=2
}

\lstset{style=mystyle}

\begin{titlepage}
  \centering
  \Large Department of Information Engineering and Computer Science
  \vfill
  \Large (DISI)
\end{titlepage}

\clearpage

\tableofcontents

\clearpage

\section{DNS specification}

\subsection{What is the DNS}

\subsection{Typical messages flow}

\subsection{DNS messages format}

\subsection{Resources representation}

\section{DNS implementation: Bind}

\subsection{Bind configuration}

\subsection{Zone files}

\section{Laboratory introduction}

\subsection{Katharà}

\subsection{Laboratory network topology}



\noindent
As always, we start with the \texttt{file} command, which gives us some information about the file we will have to analyze.

\begin{lstlisting}
 $ file behindthescenes
 behindthescenes: ELF 64-bit LSB pie executable, x86-64, version 1 (SYSV), dynamically linked, interpreter /lib64/ld-linux-x86-64.so.2, BuildID[sha1]=e60ae4c886619b869178148afd12d0a5428bfe18, for GNU/Linux 3.2.0, not stripped
\end{lstlisting}

\noindent
As we can see from the command output,
the file is an ELF executable, and not stripped! Yay!
\\\\

\noindent
Let's now try to execute the program

\begin{lstlisting}
 $ ./behindthescenes
 ./challenge <password>
\end{lstlisting}

\noindent
It asks for a password, as a command line argument. Let's try with one.

\begin{lstlisting}
 $ ./behindthescenes test
\end{lstlisting}

\noindent
It does nothing...\\
Let's try with \texttt{strings} and \texttt{ltrace}...\\
\\
\texttt{strings} gives us nothing helpful, but we see that the program uses the library functions \texttt{strlen} and \texttt{strncmp}...

\begin{lstlisting}
$ ltrace ./behindthescenes test     
  --- SIGILL (Illegal instruction) ---
  --- SIGILL (Illegal instruction) ---
  --- SIGILL (Illegal instruction) ---
  +++ exited (status 0) +++    
\end{lstlisting}

\noindent
\texttt{ltrace} here is of no help...

\noindent
Let's open it with \texttt{Ghidra}...\\
\\
NOTE: the code is full of \texttt{ud2} instructions, which is an intruction used for testing
purposes, in order to test the handling of an invalid instruction.\\
When this instruction is executed, if no handler to the \texttt{Illegal instruction} is provided, the
program will crash.\\
But in our case a handler function has been provided: \texttt{segill\_sigaction}.
So, when that instruction is executed, the program will execute that function and then resume the execution.\\
\\
The problem here is that \texttt{Ghidra} has been tricked by that instruction, so the decompiler won't be able to
work correctly, and also the disassembler will stop disassembling when that instruction is encountered.\\
So we'll have to explicitly tell \texttt{Ghidra} to disassmble more code, until the end of the main function.

\begin{lstlisting}
MOV        RAX,qword ptr [RBP + -0xb0]
ADD        RAX,0x8
MOV        RAX,qword ptr [RAX]
MOV        RDI,RAX
CALL       <EXTERNAL>::strlen                 
            ;size_t strlen(char * __s)
CMP        RAX,0xc
JNZ        LAB_00101432
UD2
MOV        RAX,qword ptr [RBP + -0xb0]
ADD        RAX,0x8
MOV        RAX,qword ptr [RAX]
MOV        EDX,0x3
LEA        RSI,[DAT_0010201b]
MOV        RDI,RAX
CALL       <EXTERNAL>::strncmp
            ;int strncmp(char * __s1, char * __s2, size_t __n)
TEST       EAX,EAX
JNZ        LAB_00101429
UD2
MOV        RAX,qword ptr [RBP + -0xb0]
ADD        RAX,0x8
MOV        RAX,qword ptr [RAX]
ADD        RAX,0x3
MOV        EDX,0x3
LEA        RSI,[DAT_0010201f]
MOV        RDI,RAX
CALL       <EXTERNAL>::strncmp
            ;int strncmp(char * __s1, char * __s2, size_t __n)
TEST       EAX,EAX
JNZ        LAB_00101420
UD2
MOV        RAX,qword ptr [RBP + -0xb0]
ADD        RAX,0x8
MOV        RAX,qword ptr [RAX]
ADD        RAX,0x6
MOV        EDX,0x3
LEA        RSI,[DAT_00102023]
MOV        RDI,RAX
CALL       <EXTERNAL>::strncmp
            ;int strncmp(char * __s1, char * __s2, size_t __n)
TEST       EAX,EAX
JNZ        LAB_00101417
UD2
MOV        RAX,qword ptr [RBP + -0xb0]
ADD        RAX,0x8
MOV        RAX,qword ptr [RAX]
ADD        RAX,0x9
MOV        EDX,0x3
LEA        RSI,[DAT_00102027]
MOV        RDI,RAX
CALL       <EXTERNAL>::strncmp
            ;int strncmp(char * __s1, char * 
TEST       EAX,EAX
JNZ        LAB_0010140e
UD2
MOV        RAX,qword ptr [RBP + -0xb0]
ADD        RAX,0x8
MOV        RAX,qword ptr [RAX]
MOV        RSI,RAX
LEA        RDI,[s_>_HTB{%s}_0010202b]  = "> HTB{%s}\n"
MOV        EAX,0x0
CALL       <EXTERNAL>::printf
            ;int printf(char * __format, ...)
\end{lstlisting}

\noindent
Here we can see that there is a call to \texttt{strlen}, which should return \texttt{0xc (=12)}
in order for the program to continue.\\
\\
After that, there are multiple calls to \texttt{strncmp}, with length 3, and with the following strings:
\begin{lstlisting}[language=c]
"Itz"
"_0n"
"Ly_"
"UD2"
\end{lstlisting}

\noindent
And that is the password.\\
\\
NOTE: the password has been divided in small parts (3 characters each), so that \texttt{strings}
won't be able to find it.\\

\end{document}